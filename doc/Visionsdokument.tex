%%%%%%%%%%%%%%%%%%%%%%%%%%%%%%%%%%%%%%%%%
% Arsclassica Article
% LaTeX Template
% Version 1.1 (10/6/14)
%
% This template has been downloaded from:
% http://www.LaTeXTemplates.com
%
% Original author:
% Lorenzo Pantieri (http://www.lorenzopantieri.net) with extensive modifications by:
% Vel (vel@latextemplates.com)
%
% License:
% CC BY-NC-SA 3.0 (http://creativecommons.org/licenses/by-nc-sa/3.0/)
%
%%%%%%%%%%%%%%%%%%%%%%%%%%%%%%%%%%%%%%%%%

%----------------------------------------------------------------------------------------
%	PACKAGES AND OTHER DOCUMENT CONFIGURATIONS
%----------------------------------------------------------------------------------------

\documentclass[
10pt, % Main document font size
a4paper, % Paper type, use 'letterpaper' for US Letter paper
oneside, % One page layout (no page indentation)
%twoside, % Two page layout (page indentation for binding and different headers)
headinclude,footinclude, % Extra spacing for the header and footer
BCOR5mm, % Binding correction
]{scrartcl}

%%%%%%%%%%%%%%%%%%%%%%%%%%%%%%%%%%%%%%%%%
% Arsclassica Article
% Structure Specification File
%
% This file has been downloaded from:
% http://www.LaTeXTemplates.com
%
% Original author:
% Lorenzo Pantieri (http://www.lorenzopantieri.net) with extensive modifications by:
% Vel (vel@latextemplates.com)
%
% License:
% CC BY-NC-SA 3.0 (http://creativecommons.org/licenses/by-nc-sa/3.0/)
%
%%%%%%%%%%%%%%%%%%%%%%%%%%%%%%%%%%%%%%%%%

%----------------------------------------------------------------------------------------
%	REQUIRED PACKAGES
%----------------------------------------------------------------------------------------

\usepackage[
nochapters, % Turn off chapters since this is an article        
beramono, % Use the Bera Mono font for monospaced text (\texttt)
eulermath,% Use the Euler font for mathematics
pdfspacing, % Makes use of pdftex’ letter spacing capabilities via the microtype package
dottedtoc % Dotted lines leading to the page numbers in the table of contents
]{classicthesis} % The layout is based on the Classic Thesis style

\usepackage{arsclassica} % Modifies the Classic Thesis package

\usepackage[T1]{fontenc} % Use 8-bit encoding that has 256 glyphs

\usepackage[utf8]{inputenc} % Required for including letters with accents

\usepackage{graphicx} % Required for including images
\graphicspath{{Figures/}} % Set the default folder for images

\usepackage{enumitem} % Required for manipulating the whitespace between and within lists

\usepackage{lipsum} % Used for inserting dummy 'Lorem ipsum' text into the template

\usepackage{subfig} % Required for creating figures with multiple parts (subfigures)

\usepackage{amsmath,amssymb,amsthm} % For including math equations, theorems, symbols, etc

\usepackage{varioref} % More descriptive referencing

%----------------------------------------------------------------------------------------
%	THEOREM STYLES
%---------------------------------------------------------------------------------------

\theoremstyle{definition} % Define theorem styles here based on the definition style (used for definitions and examples)
\newtheorem{definition}{Definition}

\theoremstyle{plain} % Define theorem styles here based on the plain style (used for theorems, lemmas, propositions)
\newtheorem{theorem}{Theorem}

\theoremstyle{remark} % Define theorem styles here based on the remark style (used for remarks and notes)

%----------------------------------------------------------------------------------------
%	HYPERLINKS
%---------------------------------------------------------------------------------------

\hypersetup{
%draft, % Uncomment to remove all links (useful for printing in black and white)
colorlinks=true, breaklinks=true, bookmarks=true,bookmarksnumbered,
urlcolor=webbrown, linkcolor=RoyalBlue, citecolor=webgreen, % Link colors
pdftitle={}, % PDF title
pdfauthor={\textcopyright}, % PDF Author
pdfsubject={}, % PDF Subject
pdfkeywords={}, % PDF Keywords
pdfcreator={pdfLaTeX}, % PDF Creator
pdfproducer={LaTeX with hyperref and ClassicThesis} % PDF producer
} % Include the structure.tex file which specified the document structure and layout

\hyphenation{Fortran hy-phen-ation} % Specify custom hyphenation points in words with dashes where you would like hyphenation to occur, or alternatively, don't put any dashes in a word to stop hyphenation altogether

%----------------------------------------------------------------------------------------
%	TITLE AND AUTHOR(S)
%----------------------------------------------------------------------------------------

\title{\normalfont\spacedallcaps{Fotos per Gesichtserkennung sortieren}} % The article title

\author{\spacedlowsmallcaps{Johannes Horvath \& Eva Witzel}} % The article author(s) - author affiliations need to be specified in the AUTHOR AFFILIATIONS block

\date{15.10.2014} % An optional date to appear under the author(s)

%----------------------------------------------------------------------------------------

\begin{document}

%----------------------------------------------------------------------------------------
%	HEADERS
%----------------------------------------------------------------------------------------

\renewcommand{\sectionmark}[1]{\markright{\spacedlowsmallcaps{#1}}} % The header for all pages (oneside) or for even pages (twoside)
%\renewcommand{\subsectionmark}[1]{\markright{\thesubsection~#1}} % Uncomment when using the twoside option - this modifies the header on odd pages
\lehead{\mbox{\llap{\small\thepage\kern1em\color{halfgray} \vline}\color{halfgray}\hspace{0.5em}\rightmark\hfil}} % The header style

\pagestyle{scrheadings} % Enable the headers specified in this block

%----------------------------------------------------------------------------------------
%	TABLE OF CONTENTS & LISTS OF FIGURES AND TABLES
%----------------------------------------------------------------------------------------

\maketitle % Print the title/author/date block

\setcounter{tocdepth}{2} % Set the depth of the table of contents to show sections and subsections only

\tableofcontents % Print the table of contents



%----------------------------------------------------------------------------------------
%	ABSTRACT
%----------------------------------------------------------------------------------------

\section*{Einleitung} 
Ziel des Programmes ist es, Gesichter auf Fotos zu erkennen und dann zu sortieren.
Die Fotos werden zuerst nach Landschaftsbilder und Fotos mit Personen sortiert. Danach werden die Gesichter der Personen mit einer Maske (Gesicht als Vorlage) abgeglichen, wobei Übereinstimmung ausgegeben werden.
Dieses Dokument sammelt dazu erste Ideen und Anforderungen.

\section{Produktbeschreibung} 
\subsection{Problemstellung}
In einem Urlaub mit Freunden entstehen meistens viele Fotos. Dabei ist es manchmal hilfreich zwischen verschiedenen Arten der Motive, wie z.B. Portrait-, Gruppenfotos oder Landschafts- bzw. Architekturaufnahmen, zu unterscheiden. Außerdem kann es hilfreich sein, die Fotos personenbezogen zu sortieren, d.h. auf den Fotos erkannte Gesichter einer bestimmten Person zuzuordnen. 

\subsection{Produktpositionierung}
Die benutzerfreundliche GUI bietet jeder Person die Möglichkeit seine Fotos nach den oben genannten Kriterien zu sortieren. Bei der personenbezogenen Sortierung werden die Gesichter mit einer Maske(Gesicht) verglichen und die Übereinstimmungen ausgegeben. Die Daten werden offline ausgewertet und greifen nicht auf online Datenbanken oder Algorithmen zu. Weiterhin wird das Produkt als Open Source zur Verfügung gestellt und der Quellcode offen gelegt.

\subsection{Zielgruppe}
Das Programm richtet sich an alle Benutzer, die die oben genannten Aspekte in den Sortiervorgang ihrer Fotos einbeziehen wollen.

\section{Anforderungen und Features}
\begin{itemize}
\item benutzerfreundliche GUI
\item verschiedene Sortieroptionen
\begin{itemize}
\item Unterscheidung der Motive
\item Unterscheidung zwischen Einzel- und Gruppenfotos
\item Bilder einzelnen Personen zuordnen
\item Gesichtsvorlage aus einer Datei oder aus einem Webcamstream
\item gefilterte Fotoauswahl entweder als Text mit Dateinamen ausgeben oder direkt kopieren
\end{itemize}
\item zuverlässige Gesichtserkennung
\item Auswertung offline
\item optional:
\begin{itemize}
\item Datenbank mit bereits erkannten Gesichtern erstellen
\item Automatische Erkennung anhand der Datenbank
\end{itemize}
\end{itemize}

\section{Plattform}
Ziel sollte ein plattformunabhängiges Produkt sein. Jedoch wird Linux (Ubuntu 14.04) wegen der OpenSource-Orientierung und der Vertrautheit des Teams mit dieser Plattform zum Entwickeln verwendet. Mit fortlaufendem Projekt soll auch die Installation vereinfacht werden.


%----------------------------------------------------------------------------------------
%	BIBLIOGRAPHY
%----------------------------------------------------------------------------------------

\renewcommand{\refname}{\spacedlowsmallcaps{References}} % For modifying the bibliography heading

\bibliographystyle{unsrt}

\bibliography{sample.bib} % The file containing the bibliography

%----------------------------------------------------------------------------------------

\end{document}